\chapter{Introduction}\label{ch:intro}

\section{Problem Statement}
"History doesn't repeat itself, but it often rhymes" - Mark Twain

The history of human civilization is soaked in blood and violence. Since the dawn of time, mankind has spent much of our existence waging wars and committing terrible acts of violence against each other. Two thousand years ago the Romans sacked their biter rivals Carthage, butchering half its inhabitants and selling the rest into slavery \cite{carthgeae}. The ancient city was razed to the ground, ensuring Roman hegemony in the Mediterranean. Since then our species evolved and so did our capacity for violence.

The industrial revolution brought on technological advancements that facilitated the execution of mass killings that have never been see before. The Nazi's systematic extermination of the Jewish populations in occupied Europe serves as a truly horrific reminder of what is possible when the resources of a highly industrialized state are coupled with a racist ideology \cite{holocuast}. Even with the end of the World War II and the total defeat of the axis powers ushering a period of increased peace and global stability, there have been numerous recurring episodes of targeted violence across the globe \cite{genocideoccur}. From the valleys of the Balkans during the Yugoslav wars to infamous killing fields of Darfur on the fringes of the Sahara during the Sudanese civil war, these episodes of violence serve as a reminders that targeted mass killings (TMKs) are not a thing of the past. 

Targeted Mass Killings are not spontaneous \cite{JasonBea}. In the lead up to these atrocities a consistent sequence of events usually occurs. This set of reoccurring patterns usually involves the systematic deterioration of socioeconomic factors that ultimately culminate in a TMK episode \cite{Harff2003}. This project seeks to forecast occurrences of TMKs by identifying the patterns of various factors in the lead up to past TMK events.


%The identification and forecasting of mass killings has come a long since Raphael Lemkin first campaigned for genocide to be codified as a international crime in the still nascent United Nations Assembly in 1948. Since then the the study of these mass killings has evolved from a tiny group of scholars to becoming an important function of the United Nations to actively prevent the occurrence of and trail those found responsible.


\section{Aim}
This project aims to assess the effectiveness of applying deep learning methods in predicting Targeted Mass Killings (TMKs)\\

\section{Hypothesis}
\begin{enumerate}
  \item Deep learning methods will be more effective than traditional linear regression techniques in predicting occurrences of TMKs. 
  \item Enriching the existing TMK dataset with additional events will improve TMK forecasting.
\end{enumerate}

\section{Motivation}
Today there are numerous social conflict databases \cite{database} that record and track episodes of tumultuous social upheaval throughout history. Additionally, there have been great improvements in deep learning prediction models. With increased data tracking and advancements in forecasting models, the prediction of TMKs is perhaps more achievable today than at any other point in human history. In recent years some work has been done to attempt to forecast occurrences of genocides that are a subset of TMKs, this work has mostly been driven by political scientists, meaning that there is scope to explore how deep learning can improve current forecasting models. This project hopes to extend on the work of Goldsmith and Butcher (2018): \emph{Genocide Forecasting Accuracy and New Forecasting to 2020} \cite{Goldsmith2018} where linear regression models were applied.

An objective of this project is to utilize deep learning techniques to predict episodes of TMKs. Up to this point the application of deep learning in social sciences is relatively rare and it is hoped that they will be a useful tool in forecasting such killings.

